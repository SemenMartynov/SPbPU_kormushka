\documentclass[a4paper,12pt]{article} %размер бумаги устанавливаем А4, шрифт 12пунктов
\usepackage[utf8]{inputenc}%включаем свою кодировку: koi8-r или utf8 в UNIX, cp1251 в Windows
\usepackage[english,russian]{babel}%используем русский и английский языки с переносами
\usepackage{amsmath} %подключаем нужные пакеты расширений
\usepackage{changepage} %пакет для отступов
\usepackage{color} %цвета

\usepackage{geometry} % Меняем поля страницы
\geometry{left=2cm}% левое поле
\geometry{right=1.5cm}% правое поле
\geometry{top=1.5cm}% верхнее поле
\geometry{bottom=1.5cm}% нижнее поле

\setcounter{tocdepth}{1}% chapter и section;
\setlength{\parindent}{0pt} % we don't want any paragraph indentation

\begin{document}
\thispagestyle{empty}% Reset page style to 'empty'

\begin{flushright}

Kormushka: финансовый менеджмент малого коллектива

Предложение 2. Web-сервис.

\hrulefill
\end{flushright}

\section*{Концепция}

Приложение является централизованным WEB-сервисом, хранящим все данные пользователей. Используя этот сервис, можно получить разнообразную статистику по тратам и предпочтениям группы, организовать какое-либо мероприятие, обменяться мнениями.

Доступ к приложению осуществляться через веб-браузер, с использованием средств защиты трафика (https). Для оформления интерфейса используются современные подходы к оформлению и удобству использования. Интеграция с внешними сервисами возможно благодаря публичному API.

Основным преимуществом приложения будет возможность доступа из любого места, где есть интернет. Обеспечение качества путём интенсивного тестирования.

\section*{Функционал}

\begin{enumerate}

\item Авторизация по паролю или через социальные сети
\item Публично доступный сервер с шифрованием трафика (https)
\item Система рейтингов и отзывов
\item Построение аналитических графиков
\item Организация сбора средств на какое-либо событие
\item REST API для взаимодействия с другими системами
\item SMS/E-mail уведомления
\item Система заказов
\item Расписание оплат регулярных покупок (оплата воды...)
\item Доска объявлений, мини-чат
\item Резервное копирование данных

\end{enumerate}

\section*{План работ}

\begin{enumerate}
{\item Отчётный период 1 (спринт 1)}

Реализация простейших операций ввода и хранения информации.

{\item Отчётный период 2 (спринт 2)}

Реализация расширенного функционала, статистики, различных сервисов.

{\item Отчётный период 3 (спринт 3)}

Работа над API, адоптация под мобильные платформы.

{\item Отчётный период 4 (спринт 4)}

Тестирование и отладка.


\end{enumerate}

\end{document}
