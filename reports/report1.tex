 \documentclass{beamer}

\usepackage[utf8]{inputenc}
\usepackage[russian]{babel}
\usepackage{cmap}

\mode<presentation> {
\usetheme{Madrid}
\setbeamertemplate{caption}[numbered]
}

\usepackage{graphicx} % Allows including images
\usepackage{booktabs} % Allows the use of \toprule, \midrule and \bottomrule in tables

\title[Технологии разработки ПО]{Kormushka:\\финансовый менеджмент малого коллектива}

\author{Black team.}
\institute[СПб ПУ]
{
Санкт-Петербургский государственный политехнический университет \\
\medskip
\textit{https://github.com/SemenMartynov/kormushka}
}
\date{\today}

\begin{document}

\begin{frame}
\titlepage
\end{frame}

\begin{frame}
\frametitle{Содержание}
\tableofcontents
\end{frame}

%------------------------------------------------
\section{Команда}
%------------------------------------------------

\begin{frame}
\frametitle{Команда: распределение ролей}

\begin{itemize}
\item Мяснов Александр (Client),
\bigskip
\bigskip
\item Дедков Сергей (Project Manager),
\medskip
\item Антон Абрамов (Lead programmer),
\medskip
\item Влад Бусаров (Software Engineer),
\medskip
\item Николай Патраков (Software Engineer),
\medskip
\item Семён Мартынов (Quality Assurance).
\end{itemize}

\end{frame}

%------------------------------------------------
\section{Постановка задачи}
%------------------------------------------------

\begin{frame}
\frametitle{Постановка задачи}

Приложение для взаиморасчётов в не большом коллективе.

\begin{itemize}
\item Накопление данных о тратах
\item Подсчёт вклада каждого участника группы
\item Формирование отчёта за какой-то период
\item Создание плана взаиморасчётов
\end{itemize}

\end{frame}

%------------------------------------------------
\section{Процессы}
%------------------------------------------------

\begin{frame}
\frametitle{Выбор стека технологий}

\begin{columns}[t]
\column{.45\textwidth} % Left column and width
Язык:
\begin{itemize}
\item PHP
\item Python
\item Ruby
\item JavaEE
\item C\#
\item СPP
\end{itemize}

\column{.5\textwidth} % Right column and width
БД:
\begin{itemize}
\item MySQL
\item SQLite
\end{itemize}
\end{columns}

\end{frame}

%------------------------------------------------

\begin{frame}
\frametitle{Выбор лицензии}

\begin{block}{GNU General Public License}
Открытое лицензионное соглашение передачи программного обеспечения в общественную собственность
\end{block}

\begin{block}{MIT}
Лицензия свободного программного обеспечения разработанная Массачусетским технологическим институтом (MIT).
\end{block}

\end{frame}

%------------------------------------------------

\begin{frame}
\frametitle{Использование инфраструктуры GitHub}

\begin{itemize}
\item Wiki
\item Issues
\item Milestones
\end{itemize}

\end{frame}

%------------------------------------------------
\section{Встреченные сложности}
%------------------------------------------------

\begin{frame}
\frametitle{Встреченные сложности}

Отсутствие достаточной функциональной спецификации!
\bigskip

Борьба с блокером:
\begin{itemize}
\item 3 письма Ицыксону В. М.
\item 2 попытки застать Ицыксона В. М. на рабочем месте
\item 2 попытки застать Мяснова А. В. (после смены заказчика)
\end{itemize}
\bigskip

Итог: одна краткая встреча с заказчиком одного представителя команды.

\end{frame}


%------------------------------------------------
\section{Демонстрация результатов}
%------------------------------------------------

\begin{frame}
\frametitle{Демонстрация достигнутых результатов}

aaa

\end{frame}

%------------------------------------------------
\section{Цели первого спринта}
%------------------------------------------------

\begin{frame}
\frametitle{План первого спринта}

К концу первого спринта мы достигнем следующих результатов:
\medskip
\begin{itemize}
\item Составим и согласуем с заказчиком техническое задание
\medskip
\item Сформируем и начнём отслеживание набора метрик для оценки проекта
\medskip
\item Подготовим инфраструктуру для Continuous integration
\medskip
\item Сформируем alpha-версию продукта
\medskip
\item Развернём продукт на публично доступном сервере
\end{itemize}

\end{frame}

%------------------------------------------------
\section{Вопросы}
%------------------------------------------------

\begin{frame}
\Huge{\centerline{Вопросы?}}
\end{frame}

%----------------------------------------------------------------------------------------

\end{document} 
