 \documentclass{beamer}

\newlength{\wideitemsep}
\setlength{\wideitemsep}{\itemsep}
\addtolength{\wideitemsep}{10pt}
\let\olditem\item
\renewcommand{\item}{\setlength{\itemsep}{\wideitemsep}\olditem}

\usepackage[utf8]{inputenc}
\usepackage[russian]{babel}
\usepackage{cmap}

\mode<presentation> {
\usetheme{Madrid}
\setbeamertemplate{caption}[numbered]
}

\usepackage{graphicx} % Allows including images
\usepackage{booktabs} % Allows the use of \toprule, \midrule and \bottomrule in tables

\title[Технологии разработки ПО]{Kormushka:\\Software testing}

\author{Black team.}
\institute[СПб ПУ]
{
Санкт-Петербургский государственный политехнический университет \\
\medskip
\textit{https://github.com/SemenMartynov/kormushka}
}
\date{\today}

\begin{document}

\begin{frame}
\titlepage
\end{frame}

\begin{frame}
\frametitle{Содержание}
\tableofcontents
\end{frame}

%------------------------------------------------
\section{Модульное тестирование (unit testing) в django}
%------------------------------------------------

\begin{frame}
\frametitle{Тестирование по степени изолированности компонентов}

\begin{block}{Модульное тестирование}
Процесс в программировании, позволяющий проверить на корректность отдельные модули исходного кода программы.
\end{block}

\begin{block}{Интеграционное тестирование}
Одна из фаз тестирования программного обеспечения, при которой отдельные программные модули объединяются и тестируются в группе.
\end{block}

\begin{block}{Системное тестирование}
Тестирование программного обеспечения (ПО), выполняемое на полной, интегрированной системе, с целью проверки соответствия системы исходным требованиям. Системное тестирование относится к методам тестирования чёрного ящика, и, тем самым, не требует знаний о внутреннем устройстве системы.
\end{block}

\end{frame}

%------------------------------------------------

\begin{frame}
\frametitle{Объект тестирования}

В проекте можно выделить следующие элементы для тестирования:
\begin{itemize}
\item Вёрстка
\item JavaScript
\item БД
\item LDAP
\item WEB-сервер
\item back-end (Python)
\end{itemize}

\end{frame}

%------------------------------------------------

\begin{frame}
\frametitle{Test-Driven Development}

TDD -- это не метод тестирования, а метод разработки!
\bigskip

\begin{enumerate}
\item сначала пишется тест, покрывающий желаемое изменение
\item затем пишется код, который позволит пройти тест
\item под конец проводится рефакторинг нового кода к соответствующим стандартам
\end{enumerate}
\bigskip

Иногда говорят о Test First Development.

\end{frame}

%------------------------------------------------
\section{Smoke testing}
%------------------------------------------------

\begin{frame}
\frametitle{Smoke testing}

Smoke Test -- минимальный набор тестов на явные ошибки. Выполняется самим программистом; не проходящую этот тест программу не имеет смысла отдавать на более глубокое тестирование.
\medskip



\end{frame}

%------------------------------------------------
\section{Покрытие кода (code coverage)}
%------------------------------------------------

\begin{frame}
\frametitle{Покрытие кода (code coverage)}



\end{frame}

%------------------------------------------------
\section{Документирование кода (Doxygen)}
%------------------------------------------------

\begin{frame}
\frametitle{Документирование кода (Doxygen)}



\end{frame}

%------------------------------------------------
\section{Вопросы}
%------------------------------------------------

\begin{frame}
\Huge{\centerline{Вопросы?}}
\end{frame}

%----------------------------------------------------------------------------------------

\end{document} 
