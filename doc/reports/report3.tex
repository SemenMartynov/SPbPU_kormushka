 \documentclass{beamer}

\newlength{\wideitemsep}
\setlength{\wideitemsep}{\itemsep}
\addtolength{\wideitemsep}{10pt}
\let\olditem\item
\renewcommand{\item}{\setlength{\itemsep}{\wideitemsep}\olditem}

\usepackage[utf8]{inputenc}
\usepackage[russian]{babel}
\usepackage{cmap}

\mode<presentation> {
\usetheme{Madrid}
\setbeamertemplate{caption}[numbered]
}

\usepackage{graphicx} % Allows including images
\usepackage{booktabs} % Allows the use of \toprule, \midrule and \bottomrule in tables

\title[Технологии разработки ПО]{Kormushka:\\финансовый микроменеджмент}

\author{Black team.}
\institute[СПб ПУ]
{
Санкт-Петербургский государственный политехнический университет \\
\medskip
\textit{https://github.com/SemenMartynov/kormushka}
}
\date{\today}

\begin{document}

\begin{frame}
\titlepage
\end{frame}

\begin{frame}
\frametitle{Содержание}
\tableofcontents
\end{frame}

%------------------------------------------------
\section{Команда}
%------------------------------------------------

\begin{frame}
\frametitle{Команда: распределение ролей}

\begin{itemize}
\item Мяснов Александр (Client),
\bigskip
\bigskip
\item Дедков Сергей (Project Manager),
\medskip
\item Антон Абрамов (Lead programmer),
\medskip
\item Влад Бусаров (Software Engineer),
\medskip
\item Николай Патраков (Software Engineer),
\medskip
\item Семён Мартынов (Quality Assurance).
\end{itemize}

\end{frame}

%------------------------------------------------
\section{Задача}
%------------------------------------------------

\begin{frame}
\frametitle{О проекте}


kormushka\\ - инструмент финансового микроменеджмента на малом преприятии.
\bigskip

Основные особенности:
\begin{itemize}
\item Интеграция с LDAP
\item Подробная статистика
\item Web-интерфейс
\item Управление с мобильного телефона
\end{itemize}

\end{frame}

%------------------------------------------------
\section{Задачи второго спринта}
%------------------------------------------------

\begin{frame}
\frametitle{Список задач второго спринта и их статус}

На второй спринт ставились следующие задачи:
\begin{itemize}
\item Мобильный клиент (Android) -- реализована работа по REST API, работа над интерфейсом в процессе.
\item Построение красивых графиков статистики -- нет
\item Полное покрытие тестами -- нет
\item Автоматически генерируемая документация -- нет
\item Deb и (возможно) rpm пакеты с нашим приложением -- да (deb)
\end{itemize}

\end{frame}

%------------------------------------------------


\begin{frame}
\frametitle{Основные блоккеры}

Основные проблемы, с которыми мы встретились:
\begin{itemize}
\item Обин из участиков команды проболел три недели.
\item Большой объем задач нынешнего семестра.
\end{itemize}


\end{frame}

%------------------------------------------------
\section{Статистика использования СКВ}
%------------------------------------------------

\begin{frame}
\frametitle{Статистика использования СКВ}

% Может убрать этот слайд

\end{frame}

%------------------------------------------------
\section{Общение с заказчиком}
%------------------------------------------------

\begin{frame}
\frametitle{Общение с заказчиком}

С заказчиком за время спринта встреч не проводилось, т.к. на данный момент фронт работ понятен и вопрос не возникает.

\end{frame}

%------------------------------------------------
\section{Демонстрация результатов}
%------------------------------------------------

\begin{frame}
\frametitle{Демонстрация достигнутых результатов}



\end{frame}

%------------------------------------------------
\section{Цели третьего спринта}
%------------------------------------------------

\begin{frame}
\frametitle{План второго спринта}

К концу второго спринта мы достигнем следующих результатов:
\medskip
\begin{itemize}
\item Мобильный клиент (Android)
\item Завершение работы над модулем статистики
\item Завершение работы над тестами и генерацией документации
\end{itemize}

\end{frame}

%------------------------------------------------
\section{Вопросы}
%------------------------------------------------

\begin{frame}
\Huge{\centerline{Вопросы?}}
\end{frame}

%----------------------------------------------------------------------------------------

\end{document} 
