\documentclass[a4paper,12pt]{article} %размер бумаги устанавливаем А4, шрифт 12пунктов
\usepackage[utf8]{inputenc}%включаем свою кодировку: koi8-r или utf8 в UNIX, cp1251 в Windows
\usepackage[english,russian]{babel}%используем русский и английский языки с переносами
\usepackage{amsmath} %подключаем нужные пакеты расширений
\usepackage{changepage} %пакет для отступов
\usepackage{color} %цвета

\usepackage{geometry} % Меняем поля страницы
\geometry{left=2cm}% левое поле
\geometry{right=1.5cm}% правое поле
\geometry{top=1.5cm}% верхнее поле
\geometry{bottom=1.5cm}% нижнее поле

\setcounter{tocdepth}{1}% chapter и section;
\setlength{\parindent}{0pt} % we don't want any paragraph indentation

\begin{document}
\thispagestyle{empty}% Reset page style to 'empty'

\begin{flushright}

Kormushka: финансовый менеджмент малого коллектива

Предложение 1. P2P сервис.

\hrulefill
\end{flushright}

\section*{Концепция}

Приложение представляет из себя программу для мобильного телефона (под все основные мобильные платформы - iOS, Android, Tizen). Пользователь может в любой момент посмотреть историю покупок, траты и оценки, внести свои изменения и дополнения. Мобильные приложения синхронизуют между собой базу данных по локальной сети (WiFi) с использованием средств криптографии.


В работе можно использовать все возможности современных мобильных телефонов, такие как геолокация, фотографирование, голосовые заметки.


Основным преимуществом приложения будет то, что данные не находятся на каком-либо публично доступном сервисе, а передаются только между непосредственными пользователями. Обеспечение качества путём интенсивного тестирования.

\section*{Функционал}

\begin{enumerate}

\item Защита данных паролем
\item Синхронизация по локальной сети (WiFi)
\item Загрузка фотографии чека и/или продукта
\item Система рейтингов и отзывов
\item Выгрузка данных в XML/PDF
\item Напоминания о долгах
\item Планировщик покупок

\end{enumerate}

\section*{План работ}

\begin{enumerate}
{\item Отчётный период 1 (спринт 1)}

Реализация локальной версии с возможностью добавлять и удалять записи.

{\item Отчётный период 2 (спринт 2)}

Реализация расширенного функционала (отзывы, фотографии).

{\item Отчётный период 3 (спринт 3)}

Реализация синхронизации по сети.

{\item Отчётный период 4 (спринт 4)}

Тестирование и отладка.


\end{enumerate}

\end{document}
