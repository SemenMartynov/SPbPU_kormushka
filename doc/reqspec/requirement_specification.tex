\documentclass[a4paper, 12pt]{article}		% general format

%%%% Charset
\usepackage{cmap}							% make PDF files searchable and copyable
\usepackage[utf8]{inputenc}					% accept different input encodings
\usepackage[T2A]{fontenc}					% russian font
\usepackage[russian]{babel}					% multilingual support (T2A)

%%%% Graphics
\usepackage[dvipsnames]{xcolor}			% driver-independent color extensions
\usepackage{graphicx}						% enhanced support for graphics
\usepackage{wrapfig}						% produces figures which text can flow around

%%%% Math
\usepackage{amsmath}						% American Mathematical Society (AMS) math facilities
\usepackage{amsfonts}						% fonts from the AMS
\usepackage{amssymb}						% additional math symbols

%%%% Typograpy (don't forget about cm-super)
\usepackage{microtype}						% subliminal refinements towards typographical perfection
\linespread{1.3}							% line spacing
\usepackage[left=2.5cm, right=1.5cm, top=2.5cm, bottom=2.5cm]{geometry}
\setlength{\parindent}{0pt}					% we don't want any paragraph indentation

%%%% Other
\usepackage{url}							% verbatim with URL-sensitive line breaks
\usepackage{listings}						% typeset source code listings
\lstset{
	breaklines=true, % Перенос длинных строк
	basicstyle=\ttfamily\footnotesize, % Шрифт для отображения кода
	frame=tblr % draw a frame at all sides of the code block
	tabsize=2, % tab space width
	showstringspaces=false, % don't mark spaces in strings
	% Настройка отображения номеров строк
	numbers=left, % Слева отображаются номера строк
	stepnumber=1, % Каждую строку нумеровать
	numbersep=5pt, % Отступ от кода
}
\renewcommand{\lstlistingname}{Листинг} % Переименование Listings в нужное именование структуры
\setcounter{secnumdepth}{5}

\title{%
\flushright
\rule{16cm}{5pt}\vskip1cm
\Huge{ТЕХНИЧЕСКОЕ ЗАДАНИЕ}\\ % на основе ГОСТ 34.602-89
\vspace{2cm}
для\\
\vspace{2cm}
информационной системы\\финансового менеджмента предприятия\\
\vspace{2cm}
\LARGE{Kormushka\\}
\vspace{2cm}
\LARGE{Версия 0.1 (draft)\\}
\vspace{2cm}
Подготовлено чёрной командой\\
\vfill
\rule{16cm}{5pt}
}
\date{}
%------------------------------------------------------------------------------

\begin{document}

\maketitle


\tableofcontents{}

%------------------------------------------------------------------------------
\newpage
\section{Общие сведения}
\subsection{Наименование системы}
\subsubsection{Полное наименование системы}
Полное наименование: Информационная система финансового менеджмента малого предприятия Kormushka.

\subsubsection{Краткое наименование системы}
Краткое наименование: ИС, Система, Kormushka.

\subsection{Основания для проведения работ}
Работа выполняется в качестве практического задания по предмету "Технологии разработки программного обеспечения"

\subsection{1.3. Наименование организаций – Заказчика и Разработчика}

\subsubsection{Заказчик}
Заказчик: Мяснов Александр Владимирович\\
Адрес фактический: г. Санкт-Петербург\\
Телефон / Факс: +7 (000) 000 00 00

\subsubsection{Разработчик}
Разработчик: Чёрная команда\\
Адрес фактический: г. Санкт-Петербург\\
Телефон / Факс: +7 (000) 000 00 00

\subsection{Плановые сроки начала и окончания работы}
Плановые сроки начала и окончания работ соответствуют продолжительности весеннего семестра.

\subsection{Источники и порядок финансирования}
Работа оплате не подлежит.

\subsection{Порядок оформления и предъявления заказчику результатов работ}
%Определяется порядок оформления и предъявления заказчику результатов работ по созданию системы (ее частей), по изготовлению и наладке отдельных средств (технических, программных, информационных) и программно-технических (программно-методических) комплексов системы.

Работы по созданию Системы сдаются Разработчиком поэтапно в соответствии с планом спринтов. По окончании каждого из этапов работ Разработчик отчитывается перед Заказчику и демонстрирует свои результаты на общем собрании всех команд.

%------------------------------------------------------------------------------
\newpage
\section{Назначение и цели создания системы}
\subsection{Назначение системы}

%Указать вид автоматизируемой деятельности (указать для управления какими процессами предназначена система).

\subsection{Цели создания системы}

%Наименования и требуемые значения технических, технологических, производственно-экономических или других показателей объекта автоматизации, которые должны быть достигнуты в результате создания АИС; критерии оценки достижения целей создания системы.

%------------------------------------------------------------------------------
\newpage
\section{Характеристика объектов автоматизации}

%Приводятся краткие сведения о сфере автоматизации с указанием ссылок на ранее разработанные документы, содержащие более подробные сведения об организации заказчика.

%------------------------------------------------------------------------------
\newpage
\section{Требования к системе}
\subsection{Требования к системе в целом}
\subsubsection{Требования к структуре и функционированию системы}

%Определяется перечень функциональных подсистем, их назначение и основные характеристики, требования к числу уровней иерархии и степени централизации системы.

\subsubsection{Требования к численности и квалификации персонала системы и режиму его работы}

\paragraph{Требования к численности персонала}

%Состав персонала, необходимого для обеспечения эксплуатации Системы в рамках соответствующих подразделений Заказчика, необходимо выделение следующих ответственных лиц:

\paragraph{Требования к квалификации персонала}

% Квалификации персонала, эксплуатирующего Систему

\paragraph{Требования к режимам работы персонала}

% Время рабочей смены

\subsubsection{Показатели назначения}

\paragraph{Параметры, характеризующие степень соответствия системы назначению\\}

Система должна обеспечивать следующие количественные показатели, которые характеризуют степень соответствия ее назначению:
\begin{itemize}
\item Количество измерений -- X.
\item Количество показателей -- Y.
\end{itemize}

\paragraph{Требования к приспособляемости системы к изменениям\\}

%Обеспечение приспособляемости системы должно выполняться за счёт...

\paragraph{Требования сохранению работоспособности системы в различных вероятных условиях\\}

В зависимости от различных вероятных условий система должна выполнять требования ...

\subsubsection{Требования к надежности}

\paragraph{Состав показателей надёжности для системы в целом\\}

Время устранения отказа должно быть следующим:
\begin{itemize}
\item ...
\item ...
\end{itemize}

Система должна соответствовать следующим параметрам:
\begin{itemize}
\item ...
\item ...
\end{itemize}

Средняя наработка на отказ Системы не должна быть меньше G часов.

\paragraph{Перечень аварийных ситуаций, по которым регламентируются требования к надёжности\\}

Под аварийной ситуацией понимается аварийное завершение процесса, выполняемого той или иной подсистемой Системы, а также "зависание" этого процесса.

При работе системы возможны следующие аварийные ситуации, которые влияют на надёжность работы системы:
\begin{itemize}
\item сбой в электроснабжении сервера;
\item сбой в электроснабжении рабочей станции пользователей системы;
\item сбой в электроснабжении обеспечения локальной сети (поломка сети);
\item ошибки Системы, не выявленные при отладке и испытании системы;
\item сбои программного обеспечения сервера.
\end{itemize}

\paragraph{Требования к надёжности технических средств и программного обеспечения\\}

К надежности оборудования предъявляются следующие требования:
\begin{itemize}
\item ...
\item ...
\end{itemize}

К надежности электроснабжения предъявляются следующие требования:
\begin{itemize}
\item ...
\item ...
\end{itemize}

\paragraph{Требования к методам оценки и контроля показателей надёжности на разных стадиях создания системы в соответствии с действующими нормативно-техническими документами.\\}

Проверка выполнения требований по надёжности должна производиться на этапе тестирования автоматизированным путём.

\subsubsection{Требования к эргономике и технической эстетике}

Подсистема формирования и визуализации отчетности данных должна обеспечивать удобный для конечного пользователя интерфейс.

К другим подсистемам предъявляются следующие требования к эргономике и технической эстетике.

\subsubsection{Требования к эксплуатации, техническому обслуживанию, ремонту и хранению компонентов системы}

Условия эксплуатации, а также виды и периодичность обслуживания технических средств Системы должны соответствовать требованиям по эксплуатации, техническому обслуживанию, ремонту и хранению, изложенным в документации завода-изготовителя (производителя) на них.

\subsubsection{Требования к защите информации от несанкционированного доступа}

\paragraph{Требования к информационной безопасности\\}

Обеспечение информационное безопасности Системы должно удовлетворять следующим требованиям:
\begin{itemize}
\item ...
\item ...
\end{itemize}

\paragraph{Требования к антивирусной защите\\}

Средства антивирусной защиты должны быть установлены на всех рабочих местах пользователей и администраторов Системы. Средства антивирусной защиты рабочих местах пользователей и администраторов должны обеспечивать:
\begin{itemize}
\item ...
\item ...
\end{itemize}

\paragraph{Разграничения ответственности ролей при доступе к отчётам и показателям\\}

Требования по разграничению доступа приводятся в виде матрицы разграничения прав.

\subsubsection{Требования по сохранности информации при авариях}

%Приводится перечень событий: аварий, отказов технических средств (в том числе - потеря питания) и т. п., при которых должна быть обеспечена сохранность информации в системе.

В Системе должно быть обеспечено резервное копирование данных.

\subsubsection{Требования к защите от влияния внешних воздействий}

%Приводятся требования к радиоэлектронной защите и требования по стойкости, устойчивости и прочности к внешним воздействиям применительно к программно-аппаратному окружению, на котором будет эксплуатироваться система.

\subsubsection{Требования по стандартизации и унификации}

% показатели, устанавливающие требуемую степень использования стандартных, унифицированных методов реализации функций (задач) системы, поставляемых программных средств, типовых математических методов и моделей, типовых проектных решений, унифицированных форм управленческих документов, установленных ГОСТ 6.10.1, классификаторов других категорий в соответствии с областью их применения, требования к использованию типовых автоматизированных рабочих мест, компонентов и комплексов.


\subsubsection{Дополнительные требования}

%Приводятся требования к оснащению системы устройствами для обучения персонала (тренажерами, другими устройствами аналогичного назначения) и документацией на них.

\subsubsection{Требования безопасности}

%В требования по безопасности включают требования по обеспечению безопасности при монтаже, наладке, эксплуатации, обслуживании и ремонте технических средств системы (защита от воздействий электрического тока, электромагнитных полей, акустических шумов и т. п.) по допустимым уровням освещенности, вибрационных и шумовых нагрузок.


\subsubsection{Требования к транспортабельности}

Система являются стационарной и после монтажа и проведения пуско-наладочных работ транспортировке не подлежат.

\subsection{Требования к функциям, выполняемым системой}

% По каждой подсистеме приводится перечень функций, задач или их комплексов (в том числе обеспечивающих взаимодействие частей системы)

\subsubsection{Подсистема сбора, обработки и загрузки данных}
\paragraph{Перечень функций, задач подлежащей автоматизации\\}

% Ведение журналов результатов сбора

\paragraph{Временной регламент реализации каждой функции, задачи\\}

\paragraph{Требования к качеству реализации функций, задач\\}

\paragraph{Перечень критериев отказа для каждой функции\\}

\subsection{Требования к видам обеспечения}
\subsubsection{Требования к математическому обеспечению}

%Для математического обеспечения системы приводятся требования к составу, области применения (ограничения) и способам использования в системе математических методов и моделей, типовых алгоритмов и алгоритмов, подлежащих разработке.

Не предъявляются.

\subsubsection{Требования к лингвистическому обеспечению}

%Для лингвистического обеспечения системы приводятся требования к применению в системе языков программирования высокого уровня, языков взаимодействия пользователей и технических средств системы, а также требования к кодированию и декодированию данных, к языкам ввода-вывода данных, языкам манипулирования данными, средствам описания предметной области (объекта автоматизации), к способам организации диалога.

\subsubsection{Требования к программному обеспечению}

%Для программного обеспечения системы приводят перечень покупных программных средств, а также требования:
%к независимости программных средств от используемых СВТ и операционной среды;
%к качеству программных средств, а также к способам его обеспечения и контроля;

\subsubsection{Требования к техническому обеспечению}

%Приводятся требования:
%1) к видам технических средств, в том числе к видам комплексов технических средств, программно-технических комплексов и других комплектующих изделий, допустимых к использованию в системе;
%2) к функциональным, конструктивным и эксплуатационным характеристикам средств технического обеспечения системы.

\subsubsection{Требования к метрологическому обеспечению}

Не предъявляются.

\subsubsection{Требования к организационному обеспечению}

К защите от ошибочных действий персонала предъявляются следующие требования:
\begin{itemize}
\item ...
\item ...
\end{itemize}

\subsubsection{Требования к методическому обеспечению}

%Приводятся требования к составу нормативно-технической документации системы (перечень применяемых при ее функционировании стандартов, нормативов, методик и т. п.).

\subsubsection{Требования к патентной чистоте}

%В требованиях по патентной чистоте указывают перечень стран, в отношении которых должна быть обеспечена патентная чистота системы и ее частей.

%------------------------------------------------------------------------------
\newpage
\section{Состав и содержание работ по созданию системы}

%Данный раздел должен содержать перечень стадий и этапов работ по созданию системы в соответствии с ГОСТ 24.601, сроки их выполнения, перечень организаций - исполнителей работ, ссылки на документы, подтверждающие согласие этих организаций на участие в создании системы, или запись, определяющую ответственного (заказчик или разработчик) за проведение этих работ.

%------------------------------------------------------------------------------
\newpage
\section{Порядок контроля и приёмки системы}

\subsection{Виды и объем испытаний системы}
Система подвергается испытаниям следующих видов:
\begin{itemize}
\item ...
\item ...
\end{itemize}

\subsection{Требования к приёмке работ по стадиям}

Требования к приёмке работ

%------------------------------------------------------------------------------
\newpage
\section{Требования к составу и содержанию работ по подготовке объекта автоматизации к вводу системы в действие}

%В разделе необходимо привести перечень основных мероприятий, которые следует выполнить при подготовке объекта автоматизации к вводу

\subsection{Технические мероприятия}

Силами Заказчика в срок до

\subsection{Организационные мероприятия}

Силами Заказчика в срок до

\subsection{Изменения в информационном обеспечении}

Для организации информационного обеспечения системы должен быть разработан и утвержден регламент подготовки и публикации данных из систем-источников.

%------------------------------------------------------------------------------
\newpage
\section{Требования к документированию}

% согласованный Разработчиком и Заказчиком перечень подлежащих разработке комплектов и видов документов, соответствующих требованиям ГОСТ 34.201-89 и НТД отрасли Заказчика

Вся документация должна быть подготовлена и передана в электронном виде (в формате pdf).

%------------------------------------------------------------------------------
\newpage
\section{Источники разработки}

%Перечисляются документы и информационные материалы (технико-экономическое обоснование, отчеты о законченных научно-исследовательских работах, информационные материалы на отечественные, зарубежные системы-аналоги и др.), на основании которых разрабатывалось ТЗ и которые должны быть использованы при создании системы.

Настоящее Техническое Задание разработано на основе следующих документов и информационных материалов:
\begin{itemize}
\item ГОСТ 24.701-86 «Надежность автоматизированных систем управления».
\item ГОСТ 21958-76 «Система "Человек-машина". Зал и кабины операторов. Взаимное расположение рабочих мест. Общие эргономические требования».
\item ГОСТ 12.1.004-91 «ССБТ. Пожарная безопасность. Общие требования».
\item ГОСТ Р 50571.22-2000 «Электроустановки зданий».
\end{itemize}

\end{document}